%% Generated by Sphinx.
\def\sphinxdocclass{report}
\documentclass[letterpaper,10pt,portuges]{sphinxmanual}
\ifdefined\pdfpxdimen
   \let\sphinxpxdimen\pdfpxdimen\else\newdimen\sphinxpxdimen
\fi \sphinxpxdimen=.75bp\relax
\ifdefined\pdfimageresolution
    \pdfimageresolution= \numexpr \dimexpr1in\relax/\sphinxpxdimen\relax
\fi
%% let collapsible pdf bookmarks panel have high depth per default
\PassOptionsToPackage{bookmarksdepth=5}{hyperref}
%% turn off hyperref patch of \index as sphinx.xdy xindy module takes care of
%% suitable \hyperpage mark-up, working around hyperref-xindy incompatibility
\PassOptionsToPackage{hyperindex=false}{hyperref}
%% memoir class requires extra handling
\makeatletter\@ifclassloaded{memoir}
{\ifdefined\memhyperindexfalse\memhyperindexfalse\fi}{}\makeatother

\PassOptionsToPackage{booktabs}{sphinx}
\PassOptionsToPackage{colorrows}{sphinx}

\PassOptionsToPackage{warn}{textcomp}

\catcode`^^^^00a0\active\protected\def^^^^00a0{\leavevmode\nobreak\ }
\usepackage{cmap}
\usepackage{fontspec}
\defaultfontfeatures[\rmfamily,\sffamily,\ttfamily]{}
\usepackage{amsmath,amssymb,amstext}
\usepackage{polyglossia}
\setmainlanguage{portuges}



        \setmainfont{Liberation Serif}
    


\usepackage[Sonny]{fncychap}
\ChNameVar{\Large\normalfont\sffamily}
\ChTitleVar{\Large\normalfont\sffamily}
\usepackage{sphinx}

\fvset{fontsize=auto}
\usepackage{geometry}


% Include hyperref last.
\usepackage{hyperref}
% Fix anchor placement for figures with captions.
\usepackage{hypcap}% it must be loaded after hyperref.
% Set up styles of URL: it should be placed after hyperref.
\urlstyle{same}

\addto\captionsportuges{\renewcommand{\contentsname}{Contents:}}

\usepackage{sphinxmessages}
\setcounter{tocdepth}{0}



\title{Suporte Senior}
\date{27 jul., 2025}
\release{0.01}
\author{Hamilton V.R}
\newcommand{\sphinxlogo}{\vbox{}}
\renewcommand{\releasename}{Versão}
\makeindex
\begin{document}

\pagestyle{empty}
\sphinxmaketitle
\pagestyle{plain}
\sphinxtableofcontents
\pagestyle{normal}
\phantomsection\label{\detokenize{index::doc}}



\chapter{Introdução}
\label{\detokenize{index:introducao}}\label{\detokenize{index:id1}}
\sphinxAtStartPar
Este documento tem como objetivo fornecer diretrizes e suporte técnico para a equipe de T.I. no auxílio ao setor de Recursos Humanos (RH) da organização. O RH desempenha um papel crucial na gestão de pessoas, processos e sistemas relacionados aos colaboradores, e a integração eficiente entre a tecnologia e as demandas do RH é essencial para o sucesso operacional.
Neste guia, abordaremos os principais tópicos, incluindo:
\begin{itemize}
\item {} 
\sphinxAtStartPar
Configuração e manutenção de sistemas utilizados pelo RH.

\item {} 
\sphinxAtStartPar
Solução de problemas comuns relacionados a ferramentas de gestão de pessoas.

\item {} 
\sphinxAtStartPar
Boas práticas para garantir a segurança e privacidade dos dados dos colaboradores.

\item {} 
\sphinxAtStartPar
Integração entre sistemas de T.I. e plataformas de RH.

\item {} 
\sphinxAtStartPar
Suporte técnico personalizado para usuários seniores do setor de RH.

\end{itemize}

\sphinxAtStartPar
Este documento foi desenvolvido para ser uma referência prática, facilitando a comunicação e a resolução de desafios técnicos de forma ágil e eficiente. Recomendamos que a equipe de T.I. consulte este material regularmente e o utilize como base para otimizar o suporte prestado ao RH.
Para dúvidas ou sugestões de melhoria, entre em contato com o responsável pela documentação ou com o líder da equipe de T.I.


\section{\sphinxstylestrong{Público\sphinxhyphen{}Alvo}:}
\label{\detokenize{index:publico-alvo}}
\begin{sphinxadmonition}{note}{Nota:}
\sphinxAtStartPar
Este Documento é destinado aos colaboradores que atuam no suporte
\end{sphinxadmonition}


\section{\sphinxstylestrong{Importância do Conteúdo}:}
\label{\detokenize{index:importancia-do-conteudo}}
\sphinxAtStartPar
Este documento possui extrema relevância para todos os profissionais que atuam na área de suporte de T.I., pois contém orientações detalhadas, descrições precisas e trilhas de conhecimento essenciais. Ele serve como um guia abrangente e indispensável para o profissional que terá a responsabilidade de gerir e manter o suporte dessa ferramenta vital para a COOPAVEL. Além disso, esta documentação é um recurso valioso tanto para instrução quanto para consulta, garantindo que todos os procedimentos e melhores práticas sejam seguidos de maneira consistente e eficiente. Através deste documento, espera\sphinxhyphen{}se que a equipe de suporte possa desempenhar suas funções com maior eficácia, segurança e confiança, contribuindo significativamente para o sucesso e a continuidade das operações da empresa.

\sphinxstepscope


\subsection{Informações Guias}
\label{\detokenize{info:informacoes-guias}}\label{\detokenize{info::doc}}

\subsubsection{Informações de Consulta Rápida}
\label{\detokenize{info:informacoes-de-consulta-rapida}}
\sphinxAtStartPar
Consulte aqui informações importantes para o fluxo de trabalho do setor de Suporte de T.I Senior.
Aqui você pode adicionar informações que são


\begin{savenotes}\sphinxattablestart
\sphinxthistablewithglobalstyle
\centering
\sphinxcapstartof{table}
\sphinxthecaptionisattop
\sphinxcaption{Informações de Acesso}\label{\detokenize{info:id3}}
\sphinxaftertopcaption
\begin{tabulary}{\linewidth}[t]{TT}
\sphinxtoprule
\sphinxstyletheadfamily 
\sphinxAtStartPar
Serviço:
&\sphinxstyletheadfamily 
\sphinxAtStartPar
Detalhes
\\
\sphinxmidrule
\sphinxtableatstartofbodyhook
\sphinxAtStartPar
Ramal Suporte
&
\sphinxAtStartPar
4444
\\
\sphinxhline
\sphinxAtStartPar
Número Empresa
&
\sphinxAtStartPar
45 3217\sphinxhyphen{}6546
\\
\sphinxhline
\sphinxAtStartPar
Ramal Coordenadora
&
\sphinxAtStartPar
Fulano: “6666”
\\
\sphinxhline
\sphinxAtStartPar
Sistema de Chamados(help Desk)
&
\sphinxAtStartPar
URL: {\color{red}\bfseries{}`}\sphinxurl{https://glpi.com.br}
\\
\sphinxhline
\sphinxAtStartPar
Acesso ao Servidor Produção
&
\sphinxAtStartPar
Endereço: \sphinxtitleref{tretrtt}
\\
\sphinxhline
\sphinxAtStartPar
Acesso ao Servidor Teste
&
\sphinxAtStartPar
Endereço: \sphinxtitleref{homologação}
\\
\sphinxhline
\sphinxAtStartPar
Wi\sphinxhyphen{}Fi
&
\sphinxAtStartPar
SSID: \sphinxtitleref{Visitantes}  Senha: \sphinxtitleref{fdgf@vel}
\\
\sphinxhline
\sphinxAtStartPar
Impressoras:
&
\sphinxAtStartPar
S1K45
\\
\sphinxhline
\sphinxAtStartPar
h6:
&
\sphinxAtStartPar
Sistema que possúi módulos de administração instalados no servidor local da EMpresa.
\\
\sphinxhline
\sphinxAtStartPar
h7:
&
\sphinxAtStartPar
Sistema que possúi Módulos nas nuvens.
\\
\sphinxbottomrule
\end{tabulary}
\sphinxtableafterendhook\par
\sphinxattableend\end{savenotes}

\sphinxstepscope


\subsection{Dúvidas Frequentes}
\label{\detokenize{duvidas:duvidas-frequentes}}\label{\detokenize{duvidas::doc}}
\sphinxAtStartPar
Várias dúvidas podem surgir no decorrer da jornada do suporte de T.I
Para isso Criamos essa sessão com dúvidas mais evidentes e frequentes.

\sphinxAtStartPar
Os módulos do Senior possuem telas em comum?


\bigskip\hrule\bigskip


\sphinxAtStartPar
É possível observar que alguns módulos do Senior possuem telas em comum, isso ocorre devido a padronização do sistema,
facilitando a navegação do usuário.
\begin{itemize}
\item {} 
\sphinxAtStartPar
Item 1

\item {} 
\sphinxAtStartPar
Item 2

\item {} 
\sphinxAtStartPar
Item 3

\end{itemize}

\noindent\sphinxincludegraphics{{plantuml-ebe6afd70b9343c5791c27ddaf27e8e99dda1e7d}.png}

\sphinxstepscope


\subsection{Visão Geral do Suporte de T.I Senior}
\label{\detokenize{visao_geral:visao-geral-do-suporte-de-t-i-senior}}\label{\detokenize{visao_geral::doc}}
\sphinxAtStartPar
O setor de Suporte de T.I Senior é responsável por atender as demandas
relacionadas ao sistema XXX, que é o sistema de gestão de pessoas
utilizado pela Empresa. O setor é responsável por atender as demandas, como atualizações, inserção de dados, treinamentos
e dúvidas relacionadas ao sistema, atendendo tanto o setor RH local como das filiais.


\subsubsection{Funções e Responsabilidades:}
\label{\detokenize{visao_geral:funcoes-e-responsabilidades}}
\sphinxAtStartPar
Entre as principais responsabilidades do setor de referente, estão:
o Atender aos chamados relacionados a atualizações de dados,
Inserção de dados, como inserir novos usuários para acesso ao
sistema;


\subsubsection{o Tirar Dúvidas com relação a processos;}
\label{\detokenize{visao_geral:o-tirar-duvidas-com-relacao-a-processos}}
\sphinxAtStartPar
o Atender e fornecer informações pertinentes a cada área;
o Treinamentos focados em cada área.


\subsubsection{Objetivos da Área:}
\label{\detokenize{visao_geral:objetivos-da-area}}
\sphinxAtStartPar
A área tem como objetivo estratégico, analisar as demandas, pesar as
consequências(positivas e negativas) do impacto de mudanças e
atualizações dentro do sistema, tendo em vista ser um setor estratégico
e de suma importância para a instituição.


\subsubsection{Interfaces com Outras Áreas:}
\label{\detokenize{visao_geral:interfaces-com-outras-areas}}
\sphinxAtStartPar
Os setores de Recrutamento, Desenvolvimento Humano Organizacional,
Admissão, Administração, RH Descentralizados e Filiais de maneira
abrangente conversam com setor Suporte T.I Senior. Para diversas
atividades que envolvem pessoas, Mudança de cargo, admissão
demissão etc.

\noindent\sphinxincludegraphics{{plantuml-b09f99eed6a03237dcf76f623cedbaa0888e4712}.png}

\sphinxstepscope


\subsection{Regras}
\label{\detokenize{regras:regras}}\label{\detokenize{regras::doc}}

\subsubsection{Regras relacionadas aos setores de tecnologia da informação}
\label{\detokenize{regras:regras-relacionadas-aos-setores-de-tecnologia-da-informacao}}

\paragraph{Introdução}
\label{\detokenize{regras:introducao}}

\subsection{Regras de Conduta}
\label{\detokenize{regras:regras-de-conduta}}\begin{enumerate}
\sphinxsetlistlabels{\arabic}{enumi}{enumii}{}{.}%
\item {} 
\sphinxAtStartPar
\sphinxstylestrong{Respeito Mútuo}: Todos os colaboradores devem tratar uns aos outros com respeito e cortesia.

\item {} 
\sphinxAtStartPar
\sphinxstylestrong{Confidencialidade}: Informações sensíveis e dados pessoais devem ser mantidos em sigilo e protegidos contra acesso não autorizado.

\item {} 
\sphinxAtStartPar
\sphinxstylestrong{Uso Adequado dos Recursos}: Os recursos de T.I devem ser utilizados exclusivamente para fins profissionais e de acordo com as políticas da empresa.

\item {} 
\sphinxAtStartPar
\sphinxstylestrong{Segurança da Informação}: Todos devem seguir as melhores práticas de segurança da informação, incluindo o uso de senhas fortes e a atualização regular de software.

\item {} 
\sphinxAtStartPar
\sphinxstylestrong{Reportar Incidentes}: Qualquer incidente de segurança ou violação das regras de conduta deve ser reportado imediatamente ao departamento de T.I.

\end{enumerate}
\begin{enumerate}
\sphinxsetlistlabels{\arabic}{enumi}{enumii}{}{.}%
\item {} 
\sphinxAtStartPar
\sphinxstylestrong{Manutenção de Sistemas}: A equipe de T.I é responsável pela manutenção e atualização de todos os sistemas e equipamentos.

\item {} 
\sphinxAtStartPar
\sphinxstylestrong{Suporte Técnico}: Fornecer suporte técnico eficiente e eficaz para todos os colaboradores.

\item {} 
\sphinxAtStartPar
\sphinxstylestrong{Gestão de Acessos}: Controlar e monitorar o acesso aos sistemas e dados da empresa.

\item {} 
\sphinxAtStartPar
\sphinxstylestrong{Backup de Dados}: Realizar backups regulares dos dados críticos para garantir a recuperação em caso de falhas.

\item {} 
\sphinxAtStartPar
\sphinxstylestrong{Treinamento e Capacitação}: Promover treinamentos regulares para os colaboradores sobre segurança da informação e uso adequado dos recursos de T.I.

\end{enumerate}

\sphinxstepscope


\subsection{Conceitos importantes}
\label{\detokenize{conceitos:conceitos-importantes}}\label{\detokenize{conceitos::doc}}
\sphinxAtStartPar
Um posto de trabalho é um local designado onde um funcionário realiza suas atividades profissionais. Este conceito abrange tanto o espaço físico quanto os recursos e ferramentas necessários para a execução das tarefas diárias.


\subsubsection{Elementos de um Posto de Trabalho}
\label{\detokenize{conceitos:elementos-de-um-posto-de-trabalho}}\begin{itemize}
\item {} 
\sphinxAtStartPar
\sphinxstylestrong{Espaço Físico}: Inclui a mesa, cadeira, iluminação e ergonomia do ambiente.

\item {} 
\sphinxAtStartPar
\sphinxstylestrong{Equipamentos}: Computadores, telefones, impressoras e outros dispositivos eletrônicos.

\item {} 
\sphinxAtStartPar
\sphinxstylestrong{Recursos}: Softwares, acesso à internet, materiais de escritório e outros suprimentos necessários.

\end{itemize}


\subsubsection{Importância de um Posto de Trabalho Adequado}
\label{\detokenize{conceitos:importancia-de-um-posto-de-trabalho-adequado}}
\sphinxAtStartPar
Um posto de trabalho bem estruturado é essencial para garantir a produtividade, conforto e bem\sphinxhyphen{}estar dos funcionários. Um ambiente de trabalho adequado pode reduzir o estresse, prevenir problemas de saúde e aumentar a eficiência das operações.

\noindent\sphinxincludegraphics{{plantuml-b14528f3824dba42485c7b5eed4bbfd7d7b0c6cf}.png}

\sphinxstepscope


\subsection{Consulta}
\label{\detokenize{Consulta:consulta}}\label{\detokenize{Consulta::doc}}
\sphinxAtStartPar
Existem informações que são de extrema importãncia para o Suporte,
saber montar uma pesquisa eficiente é fundamental para encontrar os
menus corretos e módulos corretos, sabendo disso é possível ter mais
agilidade e assertividade nas análises.

\noindent\sphinxincludegraphics{{plantuml-ebe6afd70b9343c5791c27ddaf27e8e99dda1e7d}.png}



\renewcommand{\indexname}{Índice}
\printindex
\end{document}