%% Generated by Sphinx.
\def\sphinxdocclass{report}
\documentclass[letterpaper,10pt,portuges]{sphinxmanual}
\ifdefined\pdfpxdimen
   \let\sphinxpxdimen\pdfpxdimen\else\newdimen\sphinxpxdimen
\fi \sphinxpxdimen=.75bp\relax
\ifdefined\pdfimageresolution
    \pdfimageresolution= \numexpr \dimexpr1in\relax/\sphinxpxdimen\relax
\fi
%% let collapsible pdf bookmarks panel have high depth per default
\PassOptionsToPackage{bookmarksdepth=5}{hyperref}
%% turn off hyperref patch of \index as sphinx.xdy xindy module takes care of
%% suitable \hyperpage mark-up, working around hyperref-xindy incompatibility
\PassOptionsToPackage{hyperindex=false}{hyperref}
%% memoir class requires extra handling
\makeatletter\@ifclassloaded{memoir}
{\ifdefined\memhyperindexfalse\memhyperindexfalse\fi}{}\makeatother

\PassOptionsToPackage{booktabs}{sphinx}
\PassOptionsToPackage{colorrows}{sphinx}

\PassOptionsToPackage{warn}{textcomp}

\catcode`^^^^00a0\active\protected\def^^^^00a0{\leavevmode\nobreak\ }
\usepackage{cmap}
\usepackage{fontspec}
\defaultfontfeatures[\rmfamily,\sffamily,\ttfamily]{}
\usepackage{amsmath,amssymb,amstext}
\usepackage{polyglossia}
\setmainlanguage{portuges}



        \setmainfont{Liberation Serif}
    


\usepackage[Sonny]{fncychap}
\ChNameVar{\Large\normalfont\sffamily}
\ChTitleVar{\Large\normalfont\sffamily}
\usepackage{sphinx}

\fvset{fontsize=auto}
\usepackage{geometry}


% Include hyperref last.
\usepackage{hyperref}
% Fix anchor placement for figures with captions.
\usepackage{hypcap}% it must be loaded after hyperref.
% Set up styles of URL: it should be placed after hyperref.
\urlstyle{same}

\addto\captionsportuges{\renewcommand{\contentsname}{Contents:}}

\usepackage{sphinxmessages}
\setcounter{tocdepth}{0}



\title{Suporte Senior}
\date{16 mar., 2025}
\release{0.01}
\author{Coopavel}
\newcommand{\sphinxlogo}{\vbox{}}
\renewcommand{\releasename}{Versão}
\makeindex
\begin{document}

\pagestyle{empty}
\sphinxmaketitle
\pagestyle{plain}
\sphinxtableofcontents
\pagestyle{normal}
\phantomsection\label{\detokenize{index::doc}}

\phantomsection\label{\detokenize{index:introducao}}
\noindent\sphinxincludegraphics[width=200\sphinxpxdimen]{{coopavel}.png}


\chapter{Introdução}
\label{\detokenize{index:id1}}
\sphinxAtStartPar
Este documento tem como objetivo fornecer diretrizes e suporte técnico para a equipe de T.I. no auxílio ao setor de Recursos Humanos (RH) da organização. O RH desempenha um papel crucial na gestão de pessoas, processos e sistemas relacionados aos colaboradores, e a integração eficiente entre a tecnologia e as demandas do RH é essencial para o sucesso operacional.
Neste guia, abordaremos os principais tópicos, incluindo:
\begin{itemize}
\item {} 
\sphinxAtStartPar
Configuração e manutenção de sistemas utilizados pelo RH.

\item {} 
\sphinxAtStartPar
Solução de problemas comuns relacionados a ferramentas de gestão de pessoas.

\item {} 
\sphinxAtStartPar
Boas práticas para garantir a segurança e privacidade dos dados dos colaboradores.

\item {} 
\sphinxAtStartPar
Integração entre sistemas de T.I. e plataformas de RH.

\item {} 
\sphinxAtStartPar
Suporte técnico personalizado para usuários seniores do setor de RH.

\end{itemize}

\sphinxAtStartPar
Este documento foi desenvolvido para ser uma referência prática, facilitando a comunicação e a resolução de desafios técnicos de forma ágil e eficiente. Recomendamos que a equipe de T.I. consulte este material regularmente e o utilize como base para otimizar o suporte prestado ao RH.
Para dúvidas ou sugestões de melhoria, entre em contato com o responsável pela documentação ou com o líder da equipe de T.I.


\section{\sphinxstylestrong{Público\sphinxhyphen{}Alvo}:}
\label{\detokenize{index:publico-alvo}}
\begin{sphinxadmonition}{note}{Nota:}
\sphinxAtStartPar
Este Documento é destinado aos colaboradores que atuam no suporte
SENIOR.
\end{sphinxadmonition}


\section{\sphinxstylestrong{Importância do Conteúdo}:}
\label{\detokenize{index:importancia-do-conteudo}}
\sphinxAtStartPar
Este documento possui extrema relevância para todos os profissionais que atuam na área de suporte de T.I., pois contém orientações detalhadas, descrições precisas e trilhas de conhecimento essenciais. Ele serve como um guia abrangente e indispensável para o profissional que terá a responsabilidade de gerir e manter o suporte dessa ferramenta vital para a COOPAVEL. Além disso, esta documentação é um recurso valioso tanto para instrução quanto para consulta, garantindo que todos os procedimentos e melhores práticas sejam seguidos de maneira consistente e eficiente. Através deste documento, espera\sphinxhyphen{}se que a equipe de suporte possa desempenhar suas funções com maior eficácia, segurança e confiança, contribuindo significativamente para o sucesso e a continuidade das operações da empresa.

\sphinxstepscope


\subsection{Informações Guias}
\label{\detokenize{info:informacoes-guias}}\label{\detokenize{info::doc}}

\subsubsection{Informações de Consulta Rápida}
\label{\detokenize{info:informacoes-de-consulta-rapida}}
\sphinxAtStartPar
Consulte aqui informações importantes para o fluxo de trabalho do setor de Suporte de T.I Senior.
Aqui você pode adicionar informações que são


\begin{savenotes}\sphinxattablestart
\sphinxthistablewithglobalstyle
\centering
\sphinxcapstartof{table}
\sphinxthecaptionisattop
\sphinxcaption{Informações de Acesso}\label{\detokenize{info:id1}}
\sphinxaftertopcaption
\begin{tabulary}{\linewidth}[t]{TT}
\sphinxtoprule
\sphinxstyletheadfamily 
\sphinxAtStartPar
Serviço:
&\sphinxstyletheadfamily 
\sphinxAtStartPar
Detalhes
\\
\sphinxmidrule
\sphinxtableatstartofbodyhook
\sphinxAtStartPar
Ramal Suporte Senior
&
\sphinxAtStartPar
5609
\\
\sphinxhline
\sphinxAtStartPar
Número Coopavel
&
\sphinxAtStartPar
45 3218\sphinxhyphen{}5000
\\
\sphinxhline
\sphinxAtStartPar
Ramal Coordenadora
&
\sphinxAtStartPar
Cláudia: “5081”
\\
\sphinxhline
\sphinxAtStartPar
Sistema de Chamados(help Desk)
&
\sphinxAtStartPar
URL: \sphinxtitleref{https://glpi.coopavel.com.br}
\\
\sphinxhline
\sphinxAtStartPar
Acesso ao Servidor Produção
&
\sphinxAtStartPar
Endereço: \sphinxtitleref{cooplogrem://wts\_rh.sol.com.br}
\\
\sphinxhline
\sphinxAtStartPar
Acesso ao Servidor Teste
&
\sphinxAtStartPar
Endereço: \sphinxtitleref{Estige}
\\
\sphinxhline
\sphinxAtStartPar
Wi\sphinxhyphen{}Fi
&
\sphinxAtStartPar
SSID: \sphinxtitleref{Visitantes}  Senha: \sphinxtitleref{coop@vel}
\\
\sphinxhline
\sphinxAtStartPar
Impressoras:
&
\sphinxAtStartPar
S1KY \sphinxhyphen{}ADM \sphinxhyphen{}02 E CL112 \sphinxhyphen{}COL1 \sphinxhyphen{} Colorida
\\
\sphinxhline
\sphinxAtStartPar
G5:
&
\sphinxAtStartPar
Sistema que possúi módulos de administração instalados no servidor local da coopavel.
\\
\sphinxhline
\sphinxAtStartPar
G7:
&
\sphinxAtStartPar
Sistema que possúi Módulos nas nuvens, exemplo módulo que marcação de ponto.
\\
\sphinxbottomrule
\end{tabulary}
\sphinxtableafterendhook\par
\sphinxattableend\end{savenotes}

\sphinxstepscope


\subsection{Dúvidas Frequentes}
\label{\detokenize{duvidas:duvidas-frequentes}}\label{\detokenize{duvidas::doc}}
\sphinxAtStartPar
Várias dúvidas podem surgir no decorrer da jornada do suporte de T.I
Para isso Criamos essa sessão com dúvidas mais evidentes e frequentes.

\sphinxAtStartPar
Os módulos do Senior possuem telas em comum?


\bigskip\hrule\bigskip


\sphinxAtStartPar
É possível observar que alguns módulos do Senior possuem telas em comum, isso ocorre devido a padronização do sistema,
facilitando a navegação do usuário.
\begin{itemize}
\item {} 
\sphinxAtStartPar
Item 1

\item {} 
\sphinxAtStartPar
Item 2

\item {} 
\sphinxAtStartPar
Item 3

\end{itemize}

\noindent\sphinxincludegraphics{{plantuml-ebe6afd70b9343c5791c27ddaf27e8e99dda1e7d}.png}

\sphinxstepscope


\subsection{Visão Geral do Suporte de T.I Senior}
\label{\detokenize{visao_geral:visao-geral-do-suporte-de-t-i-senior}}\label{\detokenize{visao_geral::doc}}
\sphinxAtStartPar
O setor de Suporte de T.I Senior é responsável por atender as demandas
relacionadas ao sistema Senior, que é o sistema de gestão de pessoas
utilizado pela COOPAVEL. O setor é responsável por atender as demandas, como atualizações, inserção de dados, treinamentos
e dúvidas relacionadas ao sistema, atendendo tanto o setor RH local como das filiais.


\subsubsection{Funções e Responsabilidades:}
\label{\detokenize{visao_geral:funcoes-e-responsabilidades}}
\sphinxAtStartPar
Entre as principais responsabilidades do setor de referente, estão:
o Atender aos chamados relacionados a atualizações de dados,
Inserção de dados, como inserir novos usuários para acesso ao
sistema;


\subsubsection{o Tirar Dúvidas com relação a processos;}
\label{\detokenize{visao_geral:o-tirar-duvidas-com-relacao-a-processos}}
\sphinxAtStartPar
o Atender e fornecer informações pertinentes a cada área;
o Treinamentos focados em cada área.


\subsubsection{Objetivos da Área:}
\label{\detokenize{visao_geral:objetivos-da-area}}
\sphinxAtStartPar
A área tem como objetivo estratégico, analisar as demandas, pesar as
consequências(positivas e negativas) do impacto de mudanças e
atualizações dentro do sistema, tendo em vista ser um setor estratégico
e de suma importância para a instituição.


\subsubsection{Interfaces com Outras Áreas:}
\label{\detokenize{visao_geral:interfaces-com-outras-areas}}
\sphinxAtStartPar
Os setores de Recrutamento, Desenvolvimento Humano Organizacional,
Admissão, Administração, RH Descentralizados e Filiais de maneira
abrangente conversam com setor Suporte T.I Senior. Para diversas
atividades que envolvem pessoas, Mudança de cargo, admissão
demissão etc.

\noindent\sphinxincludegraphics{{plantuml-b09f99eed6a03237dcf76f623cedbaa0888e4712}.png}

\sphinxstepscope


\subsection{Regras}
\label{\detokenize{regras:regras}}\label{\detokenize{regras::doc}}

\subsubsection{Regras relacionadas aos setores de tecnologia da informação}
\label{\detokenize{regras:regras-relacionadas-aos-setores-de-tecnologia-da-informacao}}

\paragraph{Introdução}
\label{\detokenize{regras:introducao}}

\subsection{Regras de Conduta}
\label{\detokenize{regras:regras-de-conduta}}\begin{enumerate}
\sphinxsetlistlabels{\arabic}{enumi}{enumii}{}{.}%
\item {} 
\sphinxAtStartPar
\sphinxstylestrong{Respeito Mútuo}: Todos os colaboradores devem tratar uns aos outros com respeito e cortesia.

\item {} 
\sphinxAtStartPar
\sphinxstylestrong{Confidencialidade}: Informações sensíveis e dados pessoais devem ser mantidos em sigilo e protegidos contra acesso não autorizado.

\item {} 
\sphinxAtStartPar
\sphinxstylestrong{Uso Adequado dos Recursos}: Os recursos de T.I devem ser utilizados exclusivamente para fins profissionais e de acordo com as políticas da empresa.

\item {} 
\sphinxAtStartPar
\sphinxstylestrong{Segurança da Informação}: Todos devem seguir as melhores práticas de segurança da informação, incluindo o uso de senhas fortes e a atualização regular de software.

\item {} 
\sphinxAtStartPar
\sphinxstylestrong{Reportar Incidentes}: Qualquer incidente de segurança ou violação das regras de conduta deve ser reportado imediatamente ao departamento de T.I.

\end{enumerate}
\begin{enumerate}
\sphinxsetlistlabels{\arabic}{enumi}{enumii}{}{.}%
\item {} 
\sphinxAtStartPar
\sphinxstylestrong{Manutenção de Sistemas}: A equipe de T.I é responsável pela manutenção e atualização de todos os sistemas e equipamentos.

\item {} 
\sphinxAtStartPar
\sphinxstylestrong{Suporte Técnico}: Fornecer suporte técnico eficiente e eficaz para todos os colaboradores.

\item {} 
\sphinxAtStartPar
\sphinxstylestrong{Gestão de Acessos}: Controlar e monitorar o acesso aos sistemas e dados da empresa.

\item {} 
\sphinxAtStartPar
\sphinxstylestrong{Backup de Dados}: Realizar backups regulares dos dados críticos para garantir a recuperação em caso de falhas.

\item {} 
\sphinxAtStartPar
\sphinxstylestrong{Treinamento e Capacitação}: Promover treinamentos regulares para os colaboradores sobre segurança da informação e uso adequado dos recursos de T.I.

\end{enumerate}

\sphinxAtStartPar
Este documento visa estabelecer diretrizes claras e procedimentos essenciais para o uso
seguro e eficiente dos sistemas, equipamentos e recursos de tecnologia da informação
disponibilizados pela Coopavel. É de responsabilidade de todos os colaboradores
conhecer e seguir as políticas aqui estabelecidas, garantindo assim a integridade,
confidencialidade e disponibilidade das informações corporativas.

\sphinxAtStartPar
Cada colaborador tem responsabilidades específicas no que diz respeito ao uso adequado
dos sistemas e equipamentos de tecnologia da informação. É essencial compreender e
respeitar tais atribuições para garantir a segurança e o bom funcionamento dos recursos
disponíveis.

\sphinxAtStartPar
Os procedimentos de uso dos sistemas e equipamentos devem ser seguidos rigorosamente
por todos os colaboradores, visando manter a integridade dos dados e garantir a eficiência
das operações corporativas.

\sphinxAtStartPar
As estações de trabalho devem ser utilizadas de forma responsável, seguindo as diretrizes
estabelecidas pela organização para garantir a segurança e o bom funcionamento dos
sistemas.

\sphinxAtStartPar
O uso de dispositivos móveis deve ser realizado de acordo com as políticas de segurança
da informação, garantindo a proteção dos dados corporativos mesmo em ambientes
remotos.

\sphinxAtStartPar
O correio eletrônico corporativo é uma ferramenta essencial para a comunicação interna
e externa da Coopavel, devendo ser utilizado de forma adequada e segura para proteger
informações estratégicas, confidenciais ou de interesse exclusivo da Cooperativa.

\sphinxAtStartPar
O acesso à internet deve ser feito de maneira responsável e em conformidade com as
políticas estabelecidas pela organização, visando evitar a exposição a ameaças
cibernéticas e proteger a rede corporativa.

\sphinxAtStartPar
A Intranet é uma ferramenta poderosa para promover a colaboração entre os diversos
departamentos e equipes da Coopavel e seu acesso é restrito somente aos colaboradores.

\sphinxAtStartPar
O armazenamento de documentos e informações deve ser realizado em locais seguros e
autorizados, garantindo a confidencialidade e integridade dos dados.

\sphinxAtStartPar
A rede corporativa deve ser utilizada de forma segura e responsável, seguindo as
diretrizes estabelecidas para proteger informações e evitar acessos não autorizados.

\sphinxAtStartPar
O uso das impressoras deve ser realizado de acordo com as políticas da Coopavel,
visando otimizar recursos e garantir a segurança das informações impressas.

\sphinxAtStartPar
O trabalho à distância deve ser realizado de acordo com as políticas estabelecidas pela
Cooperativa, garantindo a segurança das informações mesmo em ambientes externos.

\sphinxAtStartPar
O uso de Ferramentas de Inteligência Artificial \sphinxhyphen{} IA deve ser regulado conforme diretrizes
dispostas na Política de Utilização de Ferramentas de Inteligência Artificial.

\sphinxAtStartPar
As disposições finais deste documento reforçam a importância do cumprimento das
políticas e procedimentos estabelecidos, bem como a responsabilidade de cada
colaborador na preservação da segurança da informação.

\sphinxAtStartPar
Violações da política de segurança da informação estão sujeitas a sanções disciplinares,
conforme estabelecido pelas normas internas da Cooperativa, visando proteger os ativos
de informação e manter a integridade operacional da Coopavel.

\sphinxAtStartPar
Portanto, é fundamental que todos os colaboradores compreendam e atuem de acordo com
as diretrizes estabelecidas neste documento, contribuindo para a segurança e o sucesso da
Cooperativa.

\sphinxAtStartPar
A presente norma interna tem por objetivo definir a utilização de sistemas e equipamentos
de informática pelos usuários da Coopavel Cooperativa Agroindustrial.

\sphinxAtStartPar
2.1.1. Os sistemas instalados nos equipamentos de informática, incluindo (programas,
hardwares e softwares), só podem ser modificados, reparados, consertados e mantidos
exclusivamente pelo departamento de tecnologia da informação \sphinxhyphen{} TI.

\sphinxAtStartPar
2.1.2. O departamento de tecnologia da informação, em conjunto com os departamentos
administrativo e financeiro, irá avaliar a necessidade e viabilidade de instalação de novos
equipamentos e sistemas.

\sphinxAtStartPar
2.1.3. O Gerente ou responsável, ao lidar com o desligamento ou transferência do
colaborador usuário do setor ou unidade, deve comunicar imediatamente a área de
tecnologia da informação para que sejam realizados os procedimentos de backup e
cancelamento de acessos necessários.

\sphinxAtStartPar
2.1.4. É proibida a divulgação de informações sobre a topologia de rede, tipo de
computador, nome ou número de máquinas e/ou servidores, licenças de softwares a
terceiros, incluindo colegas de trabalho de departamentos diferentes.

\sphinxAtStartPar
2.1.5. É proibida a alteração e ou decodificação (cracking) de login (usuário e senha),
assim como partilhar área de trabalho.

\sphinxAtStartPar
2.1.6. É proibida a alteração das configurações de rede e inicialização das máquinas,
assim como qualquer modificação que possa afetar negativamente o desempenho.

\sphinxAtStartPar
2.1.7. É proibido o uso de equipamentos particulares ou de terceiros, como notebook,
netbooks, tablets, CDS, impressoras, zip drivers, dispositivos móveis de qualquer
natureza e modem ADSL na Cooperativa ou na rede interna (intranet) sem a devida
autorização e acompanhamento do departamento de tecnologia da informação.

\sphinxAtStartPar
2.2.1. Cada colaborador terá um login (usuário e senha) pessoal e intransferível, sendo
estritamente proibido divulgar essas informações a terceiros ou utilizar o login de outra
pessoa para acessar os equipamentos e programas de informática.

\sphinxAtStartPar
2.2.2. Deve\sphinxhyphen{}se garantir que o sistema seja encerrado sempre que o colaborador se afastar
de seu respectivo equipamento ou automaticamente quando passados 10 (dez) minutos de
inatividade. É de responsabilidade do usuário fechar o sistema quando se afastar do
equipamento.

\sphinxAtStartPar
2.2.3. Os sistemas e equipamentos de informática destinam\sphinxhyphen{}se exclusivamente ao uso pelo
colaborador no desempenho de suas funções profissionais em benefício da Coopavel. É
terminantemente proibido o uso para outras finalidades.

\sphinxAtStartPar
2.2.4. É proibida a instalação ou utilização de qualquer tipo de jogo, inclusive jogos locais
do Windows.

\sphinxAtStartPar
2.2.5. É proibida a instalação ou uso de qualquer tipo de software nas estações de trabalho,
sendo obrigatório que o usuário utilize apenas softwares licenciados pela Cooperativa.

\sphinxAtStartPar
2.2.6. Qualquer movimentação de equipamentos de informática, será realizada
exclusivamente pelo departamento de tecnologia da informação, mediante registro de
requisições de serviços no sistema Service Desk.

\sphinxAtStartPar
2.2.7. Todas as requisições dos usuários ao departamento de tecnologia da informação,
deverão ser realizadas por meio de registros na ferramenta de chamados Service Desk.

\sphinxAtStartPar
2.2.8. O usuário deve utilizar recursos e serviços de TI da Cooperativa de forma
profissional, ética e legal.

\sphinxAtStartPar
\sphinxstylestrong{COOPAVEL}

\sphinxAtStartPar
2.3.1. É proibido ao usuário remover arquivos do sistema operacional bem como os programas
instalados em sua estação de trabalho sem a devida autorização e orientação do departamento de tecnologia.

\sphinxAtStartPar
2.3.2. É proibido ao usuário realizar qualquer modificação na configuração de hardware dos equipamentos
de informática, incluindo a instalação ou remoção de peças internas e externas.

\sphinxAtStartPar
2.3.3. O usuário tem obrigação de zelar pela conservação dos equipamentos de informática sob sua responsabilidade,
não podendo alimentar\sphinxhyphen{}se próximo a eles.

\sphinxAtStartPar
2.3.4. É de responsabilidade do usuário manter seu equipamento de informática limpo, realizando regularmente a
limpeza externa do gabinete, teclado, mouse e monitor.

\sphinxAtStartPar
2.3.5. Não se deve ausentar da estação de trabalho sem antes encerrar a sessão do navegador (browser), bloquear a
estação de trabalho e encerrar a sessão do e\sphinxhyphen{}mail corporativo, garantindo, assim, a impossibilidade de acesso indevido por terceiros.
\begin{description}
\sphinxlineitem{2.4.1. Os Aparelhos Tablets e Smartphones de propriedade da Coopavel são dimensionados de acordo com as necessidades}
\sphinxAtStartPar
dos usuários, considerando a execução de suas atividades e alinhados às premissas da Cooperativa.

\sphinxlineitem{2.4.2. É de responsabilidade do usuário, no momento do recebimento dos aparelhos, assinar o «termo de responsabilidade»}
\sphinxAtStartPar
(anexo 01) reconhecendo as regras de uso e conservação, concordando com seu teor e assumindo as responsabilidades
previstas.

\sphinxlineitem{2.4.3. Em caso de perda, furto ou roubo do aparelho, o usuário deve comunicar o fato «imediatamente» ao seu gestor}
\sphinxAtStartPar
direto, para que as medidas cabíveis sejam tomadas.

\end{description}

\sphinxAtStartPar
2.4.4. Ao encerrar o contrato de trabalho entre Usuário e Cooperativa, é obrigatório que o usuário proceda à devolução
do aparelho, no ato da rescisão do seu contrato.
\begin{description}
\sphinxlineitem{2.4.5. É proibido aos usuários realizar downloads e instalar aplicativos em dispositivos móveis pertencentes à Coopavel}
\sphinxAtStartPar
que não estejam relacionados às atividades de trabalho exercidas na Cooperativa.

\sphinxlineitem{2.4.6. O uso de aplicativos de mensagens instantâneas (WhatsApp, Telegram, Messenger, WeChat, Skype, Google Allo, entre}
\sphinxAtStartPar
outros) instalados nos aparelhos, deve ser autorizado pela Gerência do respectivo departamento, e seu uso deve estar
vinculado diretamente ao horário de expediente e eventos. Esses aplicativos são considerados um meio de comunicação
informal da Cooperativa. Esta regra também se aplica ao uso dessas ferramentas nas estações de trabalho.

\sphinxlineitem{2.4.6.1 A Cooperativa não permite que o usuário instale ou utilize os aplicativos mencionados no item 2.4.6 em aparelho}
\sphinxAtStartPar
de uso pessoal para fins de trabalho. No entanto, essa possibilidade é facultada pela Cooperativa aos colaboradores de determinados departamentos para facilitar o exercício de suas funções, mediante autorização da Gerência do respectivo departamento. É importante ressaltar que o uso desses aplicativos não é obrigatório.

\end{description}

\sphinxAtStartPar
2.5.1. Toda comunicação oficial eletrônica da Coopavel deve ser realizada por e\sphinxhyphen{}mail institucional.
\begin{description}
\sphinxlineitem{2.5.2. O correio eletrônico (e\sphinxhyphen{}mail) é considerado uma ferramenta de trabalho, e deve ser utilizado exclusivamente}
\sphinxAtStartPar
para o desenvolvimento das atividades profissionais em benefício dos interesses da Coopavel, não sendo permitido
para uso de assuntos pessoais.

\end{description}

\sphinxAtStartPar
2.5.3. É proibido ao usuário enviar e\sphinxhyphen{}mails em massa ou e\sphinxhyphen{}mails não solicitados (spam/spammers).
\begin{description}
\sphinxlineitem{2.5.4. O usuário não deve utilizar o correio eletrônico corporativo para enviar e\sphinxhyphen{}mails com propósitos comerciais,}
\sphinxAtStartPar
religiosos, políticos ou quaisquer outros que não estejam relacionados aos interesses da Cooperativa.

\sphinxlineitem{2.5.5. É estritamente proibido ao usuário enviar e\sphinxhyphen{}mails que contenham comentários ofensivos, obscenos, difamatórios}
\sphinxAtStartPar
ou qualquer material ou informação que possa prejudicar a imagem pública ou causar constrangimento à Cooperativa.

\end{description}

\sphinxAtStartPar
2.5.6. Para evitar exceder o limite máximo do espaço em disco e o acúmulo de arquivos desnecessários no servidor, é
responsabilidade do usuário realizar a manutenção da sua caixa de e\sphinxhyphen{}mail, excluindo periodicamente e\sphinxhyphen{}mails
desnecessários, bem como e\sphinxhyphen{}mails já lidos e não mais úteis.

\sphinxAtStartPar
2.5.7. É proibido ao usuário abrir e\sphinxhyphen{}mails e executar anexos com as extensões .bat, .exe, .src, .lnk, .com,
.pif, .vbs, .msi, .cmd, .jar, bem como extensões desconhecidas ou acessar links que direcionem para arquivos
com essas extensões.

\sphinxAtStartPar
2.5.8. É proibido ao usuário enviar, transmitir, manusear ou disseminar informações sigilosas, segredos de negócio
ou qualquer outra informação confidencial da Cooperativa a terceiros.
\begin{description}
\sphinxlineitem{2.5.9. É proibido o uso de endereços de e\sphinxhyphen{}mail corporativo para cadastros em sites externos, incluindo, mas não se}\begin{description}
\sphinxlineitem{limitando a compras online, download de e\sphinxhyphen{}books, cadastros em redes sociais, entre outros serviços não relacionados}
\sphinxAtStartPar
ao trabalho.

\end{description}

\sphinxlineitem{2.5.10. O departamento de tecnologia da informação reserva\sphinxhyphen{}se o direito de auditar, a qualquer momento e sem aviso}
\sphinxAtStartPar
prévio, o conteúdo das caixas postais no servidor de correio de e\sphinxhyphen{}mails.

\end{description}

\sphinxAtStartPar
2.6.1. O acesso à internet deve ser utilizado exclusivamente no desenvolvimento da atividade profissional em prol dos interesses da Cooperativa, não sendo permitido para assuntos pessoais.

\sphinxAtStartPar
2.6.2. É estritamente proibido ao usuário, acessar, visualizar, criar, postar, carregar ou encaminhar quaisquer arquivos ou mensagens de conteúdos abusivos, obscenos, insultuosos, pornográficos, ofensivos, difamatórios, agressivos, ameaçadores, vulgares, racistas, de apologia ao uso de drogas, de incentivo à violência ou outro material que possa violar qualquer lei aplicável, bem como causar constrangimento público à Cooperativa.

\sphinxAtStartPar
2.6.3. É proibido ao usuário acessar salas de bate\sphinxhyphen{}papo (chatrooms), exceto se esse acesso for necessário para a realização das atividades do departamento.

\sphinxAtStartPar
2.6.4. O usuário que possuir autorização para usar aplicativos de mensagens instantâneas (WhatsApp, Telegram, Messenger, WeChat, Skype, Google Allo, Snapchat, entre outros), deverá utilizá\sphinxhyphen{}los exclusivamente para fins relacionados aos interesses da Cooperativa.

\sphinxAtStartPar
2.6.5. É proibido ao usuário, utilizar programas para download de arquivos de compartilhamento por tópico (torrentes), bem como quaisquer outras ferramentas para baixar músicas, vídeos ou jogos.

\sphinxAtStartPar
2.6.6. É proibido ao usuário fazer download de quaisquer arquivos sem autorização do departamento de tecnologia da informação.

\sphinxAtStartPar
2.6.7. É proibido ao usuário se utilizar de ferramentas ou programas que contornem a segurança para acessar serviços não autorizados.

\sphinxAtStartPar
2.6.8. É proibido ao usuário acessar sites de programas de TV na internet ou a qualquer conteúdo sob demanda (streaming \sphinxhyphen{} distribuição de conteúdo digital pela internet).

\sphinxAtStartPar
2.6.9. É proibido ao usuário acessar sites de jogos online, bem como realizar atividades não relacionadas ao trabalho, como por exemplo micro serviços.

\sphinxAtStartPar
2.7.1. O portal e a intranet da Coopavel, bem como, as informações neles disponibilizadas, observarão as seguintes diretrizes:

\sphinxAtStartPar
2.7.1.1. As informações disponibilizadas no portal da intranet devem ser estritamente pertinentes ao interesse direto da Cooperativa. Através deste portal, os usuários têm acesso a uma variedade de link’s de acesso a sistemas e informações específicas da Coopavel, contribuindo para uma comunicação direcionada e eficaz.

\sphinxAtStartPar
2.7.1.2. O idioma utilizado será exclusivamente o português.

\sphinxAtStartPar
2.7.1.3. O conteúdo será estruturado de forma a priorizar a informação, a transparência e o fornecimento de serviços aos usuários.

\sphinxAtStartPar
2.7.1.4. O acesso ao Portal da Intranet é restrito aos colaboradores da Cooperativa, sendo proibido o compartilhamento de credenciais de acesso com terceiros.

\sphinxAtStartPar
2.8.1. O usuário deve manter sigilo sobre dados¹, documentos e informações considerados estratégicos, confidenciais ou de interesse exclusivo da Cooperativa.

\sphinxAtStartPar
2.8.2. Todos os documentos e informações considerados estratégicos ou confidenciais devem ser armazenados nos diretórios pessoais do servidor de arquivos «T», em suas pastas devidamente identificada, garantindo backup regular desses conteúdos.

\sphinxAtStartPar
2.8.3. É proibido ao usuário guardar documentos importantes ou confidenciais no diretório C:ou área de trabalho do computador Windows, devido ao risco de perdê\sphinxhyphen{}los a qualquer momento.

\sphinxAtStartPar
2.8.4. É estritamente proibido ao usuário armazenar arquivos de música, vídeos que não sejam para atividades profissionais, bem como de conteúdo pornográfico, obsceno, fraudulento, difamatório e racialmente ofensivo. O departamento de tecnologia da informação reserva\sphinxhyphen{}se o direito de excluir imediatamente sem prévia autorização do usuário, todo e qualquer material mencionado acima encontrado na rede ou localmente na estação de trabalho.

\sphinxAtStartPar
2.8.5. É estritamente proibida a utilização de dispositivos de armazenamento, como pendrives, micro drives, memory sticks, discos rígidos removíveis/externos, câmeras fotográficas, celulares smartphones, gravadores externos e/ou similares sem a autorização prévia do departamento de tecnologia da informação.

\sphinxAtStartPar
2.8.6. É estritamente proibida a utilização de armazenamento de dados na nuvem (Google Drive, Dropbox, OneDrive, entre outros) ou disco virtual (Google Workspace, Microsoft SharePoint, entre outros) sem a autorização da gerência ou diretoria.

\sphinxAtStartPar
2.8.7. É estritamente proibido gravar e fotografar os ambientes de trabalho, bem como, salvar o «print» ou «screenshots» (capturas de tela) de imagem do computador ou de dispositivos móveis de qualquer natureza, conforme disposto no item 2.8.1.

\sphinxAtStartPar
2.9.1. É estritamente proibido realizar tentativas de obter acesso não autorizado, como por exemplo, tentar fraudar a segurança de qualquer servidor, rede ou conta.

\sphinxAtStartPar
2.9.2. É proibido ao usuário, acessar diretórios ou pastas de outros usuários, a menos que tenham autorização expressa para fazê\sphinxhyphen{}lo, mesmo que esses diretórios ou pastas estejam disponíveis na rede.

\sphinxstepscope


\subsection{Conceitos importantes}
\label{\detokenize{conceitos:conceitos-importantes}}\label{\detokenize{conceitos::doc}}
\sphinxAtStartPar
Um posto de trabalho é um local designado onde um funcionário realiza suas atividades profissionais. Este conceito abrange tanto o espaço físico quanto os recursos e ferramentas necessários para a execução das tarefas diárias.


\subsubsection{Elementos de um Posto de Trabalho}
\label{\detokenize{conceitos:elementos-de-um-posto-de-trabalho}}\begin{itemize}
\item {} 
\sphinxAtStartPar
\sphinxstylestrong{Espaço Físico}: Inclui a mesa, cadeira, iluminação e ergonomia do ambiente.

\item {} 
\sphinxAtStartPar
\sphinxstylestrong{Equipamentos}: Computadores, telefones, impressoras e outros dispositivos eletrônicos.

\item {} 
\sphinxAtStartPar
\sphinxstylestrong{Recursos}: Softwares, acesso à internet, materiais de escritório e outros suprimentos necessários.

\end{itemize}


\subsubsection{Importância de um Posto de Trabalho Adequado}
\label{\detokenize{conceitos:importancia-de-um-posto-de-trabalho-adequado}}
\sphinxAtStartPar
Um posto de trabalho bem estruturado é essencial para garantir a produtividade, conforto e bem\sphinxhyphen{}estar dos funcionários. Um ambiente de trabalho adequado pode reduzir o estresse, prevenir problemas de saúde e aumentar a eficiência das operações.

\noindent\sphinxincludegraphics{{plantuml-f7e193f75813de2d78ff85f0b329ed59f1e457a5}.png}

\sphinxstepscope


\subsection{Consulta}
\label{\detokenize{Consulta:consulta}}\label{\detokenize{Consulta::doc}}
\sphinxAtStartPar
Existem informações que são de extrema importãncia para o Suporte,
saber montar uma pesquisa eficiente é fundamental para encontrar os
menus corretos e módulos corretos, sabendo disso é possível ter mais
agilidade e assertividade nas análises.

\noindent\sphinxincludegraphics{{plantuml-ebe6afd70b9343c5791c27ddaf27e8e99dda1e7d}.png}



\renewcommand{\indexname}{Índice}
\printindex
\end{document}